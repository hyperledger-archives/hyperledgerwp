Hyperledger Sawtooth (\url{https://github.com/hyperledger/sawtooth-core} and further repositories named like \url{sawtooth-*}) is a modular platform for building, deploying, and running distributed ledgers. The Sawtooth design philosophy targets keeping distributed ledgers distributed and making smart contracts safe - particularly for enterprise use.

In fitting with this enterprise focus, Sawtooth is also highly modular. This enables enterprises and consortia to make policy decisions that they are best equipped to make.

Originally, Sawtooth was designed to explore scalability, security, and privacy questions prompted by the original distributed ledgers. That mandated a certain modularity that was lacking at the time. Starting from scratch allowed us to employ lessons from those pioneering systems and branch into usages that the original currency ledgers weren’t intended to address. PoET, the new consensus hits scalability, while Transaction Families, our contract logic, narrow the attack surface for contracts while simultaneously broadening the functionality. We also have a keen interest in trusted execution environments and what role that can play in private transactions.

In branching into new business cases, we felt it was important that the system preserve certain tenants of a distributed ledger. That is, in an enterprise deployment, the concept of a distributed ledger shouldn’t collapse into a replicated database. Enterprise participants need autonomy and they have the right to run their own nodes. The set of participants will also be dynamic and the system – particularly consensus – must accommodate that volatility. It is not clear, for example, whether an O(n2) protocol with fixed membership like PBFT can support the scale or volatility of a distributed ledger at production levels. Further it seems inadvisable to sidestep the challenges of providing Byzantine Fault Tolerance and operate on only a Crash Fault Tolerant consensus. Finally, we observed that, “public” and “private” define a spectrum of authorization policies – not a binary option for a distributed ledger.
