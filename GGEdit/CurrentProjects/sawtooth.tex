\textbf{Hyperledger Sawtooth} is a modular platform for building, deploying, and running distributed ledgers. 
Distributed ledgers provide a digital record (such as asset ownership) that is maintained without a central authority or implementation. 
Sawtooth aims to keep distributed ledgers distributed, and to make smart contracts safe for enterprise use.
In fitting with this focus, Sawtooth is highly modular. 
This enables enterprises and consortiums to make decisions about their blockchain applications for themselves.

\subsubsection{Technical innovations in Sawtooth}
Sawtooth contains several technical innovations, including:
\begin{itemize}
\item \textbf{Unpluggable consensus}---Going beyond compile-time pluggable consensus, this allows a consortium to change consensus algorithms on a running blockchain simply by issuing a transaction.
\item \textbf{Proof of Elapsed Time (PoET)}---A consensus algorithm with the scalability of Proof of Work but without the drawback of high power consumption.
\item \textbf{Transaction families}---A smart contract abstraction that allows users to write smart contract logic in the language of their choosing.
\item \textbf{Compatability with Ethereum contracts}---Transaction families can also integrate other smart contract interpreters including Hyperledger Burrow's Ethereum Virtual Machine. 
Sawtooth features like permissioning and un-pluggable consensus enable Ethereum to be configured appropriately for an enterprise.
\item \textbf{Parallel transaction execution}---Most blockchains require transactions to be executed in series   to guarantee consistent ordering at each peer. 
Sawtooth includes an advanced parallel scheduler that splits blocks into parallel flows. 
Parallelism allows for faster block processing to partially address the performance drawback of blockchains compared to traditional databases.
\item \textbf{Private transactions}---Clusters of Sawtooth nodes can be easily deployed with separate permissioning. 
This provides privacy and confidentiality among participants of that distinct chain. 
No centralized service leak transaction patterns or other confidential information.
However, an intermediary such as \textbf{Hyperledger Quilt} is required to connect separate chains. 
In the future, Sawtooth plans to provide additional privacy and confidentiality features on top of trusted execution environments and/or zero knowledge primitives.
\end{itemize}

\subsubsection{Sawtooth extends the earlier distributed ledgers}
Originally, Sawtooth was designed to explore scalability, security, and privacy questions prompted by the earliest distributed ledgers. 
That required a modularity that was lacking at the time. 
Starting from scratch enabled the project to draw lessons from those pioneering systems, and then extend into further use cases that the original currency ledgers weren't intended to address. 

The consensus model PoET boosts scalability.
Transaction families broaden the scope of smart contracts, while narrowing the potential attack surface. 
The Sawtooth designers are also exploring trusted execution environments and the role those can play in private transactions.

Even when branching into new business cases, certain key features of a distributed ledger must be preserved. 
In an enterprise deployment, the distributed ledger must not devolve into nothing more than a replicated database. 
Enterprise participants need autonomy and have the right to run their own nodes. 

Since the set of participants will  be dynamic, the system---particularly the consensus model---must accommodate that volatility. 

\subsubsection{Over-complex and under-explained technical details}
It is not clear, for example, whether an O(n2) protocol (should we  say this, when many readers won't have a clue what this means?---GG) with fixed membership like Practical Byzantine Fault Tolerance (PBFT) can support the scale or volatility of a distributed ledger at production levels. 
It seems unwise to sidestep the challenges of providing Byzantine Fault Tolerance to operate with only a Crash Fault Tolerant consensus. 
Finally, both public and private distributed ledgers define a spectrum of authorization policies---not a binary either/or option.

To find out more about Hyperledger Sawtooth, see \url{https://github.com/hyperledger/sawtooth-core} and further repositories named like \url{sawtooth-*}.

