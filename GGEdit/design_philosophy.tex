Distributed ledgers can have vastly different requirements for different use cases. 
For instance, when participants share high trust---such as between financial institutions---blockchains can add blocks to the chain using rapid consensus with short confirmation times. 
On the other hand, when there is minimal trust between participants, they must tolerate slower processing.

Hyperledger embraces the full spectrum of use cases. 
We recognize that different enterprise scenarios have different requirements for confirmation times, decentralization, trust, and other issues. 
And each issue represents a potential ``optimization point" for the technology. 

To address this diversity, all Hyperledger projects follow the same design philosophy. All our projects must be:  
\begin{itemize}
\item Modular  
\item Highly secure 
\item Interoperable
\item Token-agnostic 
\item Complete with APIs
\end{itemize}

\subsection{Modular} 
Hyperledger is developing modular, extensible frameworks with common building blocks that can  be reused.
This modular approach enables developers to experiment with different types of components as they evolve, and to change individual components without affecting the rest of the system. 
And this helps developers to create components that can be combined to build distributed ledger solutions well-suited to different requirements. 

This modular approach means a diverse community of developers can work independently on different modules, and re-use common modules across multiple projects. 

The Hyperledger Architecture Working Group defines functional modules and interfaces for issues such as communication, consensus, cryptography, identity, ledger storage, smart contract, and policy.\footnote{insert link to relevant webpage or doc} 

\subsection{Highly secure}
Security is a key consideration for distributed ledgers, since many use cases involve high-value transactions or sensitive data.
With large codebases, many networked nodes, and valuable data flows, distributed ledgers have become prime targets for online attackers. 
Securing a blockchain is quite a difficult task: Distributed ledgers must provide a large set of features and functions, while resisting persistent adversaries. 

Security and robustness are the keys to enable enterprise-class blockchains to evolve, and provide the critical infrastructure for next-generation business networks. 
Hyperledger projects embrace security by design and follow the best practices specified by the Linux Foundation's Core Infrastructure Initiative\footnote{insert link to relevant webpage or doc}.
And all Hyperledger algorithms, protocols, and cryptography are reviewed and audited by security experts.

\subsection{Interoperable} 
In the future, many different blockchain networks will need to communicate and exchange data to form more complex and powerful networks. 
At Hyperledger, we believe that all smart contracts and applications must be portable across many different blockchain networks. 
This high degree of interoperability will help meet the promise of blockchain technologies. 

\subsection{Token-agnostic}
Hyperledger projects are independent and agnostic of all alt-coins, crypto-currencies, and tokens. 
Hyperledger will never issue its own crypto-currency; this is decidedly not our purpose. 
And no Hyperledger project will require a native token to provide incentives to operate the network or to manage resources. 
The design philosophy includes an explicit focus on managing digital objects, which may represent currencies. 
This removes any need for crypto-currencies or native tokens.

\subsection{Complete with APIs}
All Hyperledger projects provide rich and easy-to-use APIs. 
A well-defined set of APIs and SDKs enable external clients and applications to interface quickly and easily with core DLT infrastructure. 
These APIs support the growth of a rich developer ecosystem, and help blockchain and distributed ledger technologies proliferate across a wide range of industries and use cases.
