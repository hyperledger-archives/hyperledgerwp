Distributed ledgers can have vastly different requirements for different use cases. 
For instance, when participants share high levels of trust---such as between financial institutions with legal agreements---blockchains can add blocks to the chain with shorter confirmation times due to using a more rapid consensus algorithm. 
On the other hand, when there is minimal trust between participants, they must tolerate slower processing for added security.

Hyperledger embraces the full spectrum of use cases. We recognize that different enterprise scenarios have different requirements for confirmation times, decentralization, trust, and other issues, where each issue represents a potential ``optimization point" for the technology. 

To address this diversity, all Hyperledger projects follow the same design philosophy. All our projects must be:  
\begin{itemize}
\item Modular  
\item Highly Secure 
\item Interoperable
\item Cryptocurrency-Agnostic 
\item Complete with APIs
\end{itemize}

\subsection{Modular} 
Hyperledger is developing modular, extensible frameworks with common building blocks that can  be reused.
This modular approach enables developers to experiment with different types of components as they evolve, and to change individual components without affecting the rest of the system. 
As a result, this helps developers create components that can be combined to build distributed ledger solutions well-suited to different requirements. 

This modular approach also means a diverse community of developers can work independently on different modules, and re-use common modules across multiple projects. 

The Hyperledger Architecture Working Group defines functional modules and interfaces for issues such as communication, consensus, cryptography, identity, ledger storage, smart contract, and policy.\footnote{insert link to relevant webpage or doc} 

\subsection{Highly Secure}
Security is a key consideration for distributed ledgers, especially since many use cases involve high-value transactions or sensitive data.
With large codebases, many networked nodes, and valuable data flows, distributed ledgers have become prime targets for online attackers. 
Securing a blockchain is quite a difficult task: Distributed ledgers must provide a large set of features and functions, while resisting persistent adversaries. 

Security and robustness are the keys to enable enterprise-class blockchains to evolve, and provide the critical infrastructure for next-generation business networks. 
Hyperledger projects embrace security by design and follow the best practices specified by the Linux Foundation's Core Infrastructure Initiative\footnote{insert link to relevant webpage or doc}.
As such, all Hyperledger algorithms, protocols, and cryptography are reviewed and audited by security experts as well as the wider open source community on a regular basis.

\subsection{Interoperable} 
In the future, many different blockchain networks will need to communicate and exchange data to form more complex and powerful networks. 
At Hyperledger, we believe that all smart contracts and applications must be portable across many different blockchain networks. 
This high degree of interoperability will help meet the increased adoption of blockchain and distributed ledger technologies. 

\subsection{Cryptocurrency-Agnostic}
Hyperledger projects are independent and agnostic of all cryptocurrencies and will never issue its own cryptocurrency; this is decidedly not our purpose. 
Additionally, no Hyperledger project will require a native token to provide incentives to operate the network or to manage resources. 
Howerver, the design philosophy includes the capability for tokenization that can be used for managing digital objects as well as represent currencies, depending on the purpose of the implementation, though it is not required for the network to operate.

\subsection{Complete with APIs}
All Hyperledger projects provide rich and easy-to-use APIs to allow for interoperability with other systems for easier implementation. 
A well-defined set of APIs and SDKs enable external clients and applications to interface quickly and easily with core DLT infrastructure. 
These APIs support the growth of a rich developer ecosystem, and help blockchain and distributed ledger technologies proliferate across a wide range of industries and use cases.
