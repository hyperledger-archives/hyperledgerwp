In this paper, we explained the rationale behind the creation of Hyperledger and our goals for the project. We outlined why we think an open source umbrella organization seems to be the optimal governing arrangement for a general blockchain consortium and showcased some of the many use cases that inspired our members to join and work on Hyperledger. We also described some of the features required to build effective blockchains for some of these use cases and we briefly summed up all the Hyperledger projects and where they stand at publication date. 

We know there's a lot of work left to be done and we realize that Hyperledger will probably always be a work-in-progress. 
But, perhaps with your help, we can all work together to build secure, efficient, and reliable blockchain solutions that make a difference to everyone's future.

\subsection{Introductory Sources on Blockchain}
~\newline
\emph{Blockchain Technology Overview}. National Institute of Standards and Technology, U.S. Department of Commerce. January 2018. One of the best introductions to blockchain we have seen, written in plain English while maintaining nuances. Includes a glossary of terms and acronyms. 
~\newline
~\newline
\emph{Blockchain Basics: Glossary and use cases}. IBM Developer-Works. Updated August 21, 2017. A solid explanation of blockchain terms aimed at developers new to the space. 
~\newline
~\newline
\emph{Blockchain Basics: A Non-Technical Introduction in 25 Steps}. Daniel Drescher. Apress. March 2017. Explains blockchain concepts with analogies, metaphors, and pictures, not mathematical formulas or program code. (\emph{GG ordered but hasn't received it yet, supposed to be good explanations in plain English...})
~\newline
~\newline
\emph{Blockchain Revolution: How the Technology Behind Bitcoin Is Changing Money, Business, and the World}. Don Tapscott and Alan Tapscott. Portfolio--Penguin. May 2016. Less about the technology and more about the implications of blockchain for business. The authors also have introductory videos available on YouTube. 

\subsection{Further Resources from Hyperledger}
We encourage you to use these Hyperledger resources to find more information on any blockchain-related topics you find interesting. Here are some of these resources to help you get started. 

\emph{The Hyperledger Vision} is a slide deck that sums up some blockchain 101-type information and the founding vision for Hyperledger, available at \url{https://www.hyperledger.org/resources/publications}. 
 
\emph{The Hyperledger Wiki} contains a wealth of technical information, available at \url{https://wiki.hyperledger.org/start}.

Each of the main projects under Hyperledger is working on a \emph{Getting Started} guide and has further information available at these links:
\begin{itemize}
\item Fabric---\url{https://www.hyperledger.org/projects/fabric}
\item Sawtooth---\url{https://www.hyperledger.org/projects/sawtooth}
\item Iroha---\url{https://www.hyperledger.org/projects/iroha}
\item Burrow---\url{https://www.hyperledger.org/projects/hyperledger-burrow}
\item Indy---\url{https://www.hyperledger.org/projects/hyperledger-indy}
\item Cello---\url{https://www.hyperledger.org/projects/cello}
\item Composer---\url{https://www.hyperledger.org/projects/composer}
\item Explorer---\url{https://www.hyperledger.org/projects/explorer}
\item Quilt---coming soon \url{}
\end{itemize} 

The Hyperledger Working Groups have many great technical resources, and are open to anyone with an interest in their subjects. For example, the Architecture Working Group has substantial documentation on the fundamentals of permissioned blockchain. If you're looking to explore technical details, that group is a great resource. The application-specific working groups are also great places to learn. For instance, the Identity Working Group has spent a lot of time discussing and documenting how blockchain can enable identity solutions. 

We hope that reading this paper is just the beginning of the Hyperledger experience for you. 
