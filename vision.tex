We already live in a highly interconnected world.  In the future, we believe that the world will become even more closely tied together:  more data sharing, more communication, and more digital content will become the norm in both business and personal lives.  This will necessitate careful management of security, privacy, and trust.  We view distributed ledger technology as the solution to the problem where someone needs a distributed database for which there is no single owner that is trusted by all of the users.  Thus, as interconnection increases, we believe that blockchain and DLT will become quite prevalent in society as distributed ledgers replace some (but not all) traditional databases.

This prevalence of distributed ledgers will not come about without difficulty, however.  Nothing in this space is `for free'--for instance, if you want a great deal of security and privacy features in a blockchain, you will often pay the price in terms of performance.  This implies that we will need to have a large variety of different blockchains--no one blockchain will work for all applications--that can communicate and interact seamlessly.

Thus, our long-term vision for Hyperledger is driven by two main architectural concerns:  modularity and interoperability.  We hope that, eventually, Hyperledger consists of lots of modules for various different blockchain components that can be put together by a non-expert into a cohesive, functional, and secure distributed ledger.  All of these modules would be interchangeable with other modules of the same time, and able to communicate with all other tangential modules of different types (or the same type if involved in communication between separate blockchains).  This would ideally enable a non-expert to set up a interoperable, secure Hyperledger instance quickly, easily, and efficiently.  

We want to specifically point out that we do not believe Hyperledger should be the `one distributed ledger to rule them all'.  The technical community of Hyperledger sees merit in many diverse blockchains, and, while we hope that other developers consider interoperability with Hyperledger, we do not intend for Hyperledger to be the only perimssioned distributed ledger platform.

<Merit as research platform?>

In the future, we hope that Hyperledger can solve most of the common problems in the distributed ledger space.  This will necessitate a good community, strong industrial support, and solid design principles, and, as we have hopefully illustrated in this paper, we have structured the Hyperledger Project with these tenets in mind.  It is now up to us to go out and accomplish this.