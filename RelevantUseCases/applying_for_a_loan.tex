Financial institutions want to lend, but only if a borrower is a good risk. This motivates them to gather detailed, personally-identifying information (PII) from each applicant. Regulation may demand that certain PII be shared (e.g., to prevent money laundering); on the other hand, holding too much information exposes them to risks from privacy regulation and hacking.

Loans aren’t fun for the borrower, either. They’d like to know which financial institution offers the best terms, but the application process is arduous and intrusive, and each new application they submit multiplies the effort and the risk that their data will be abused.

The identity solution offered by Hyperledger Indy is transformative here. Applicants can share just the information that the financial institution needs to make a proposal, and they can do it in a way that guarantees truth, builds lender confidence, and satisfies regulatory pressures. They can do this with a hundred different potential lenders at a time, in milliseconds, all without creating correlation risk or placing sensitive personal info in a hackable database with custodians of uncertain reputation. Instead of disclosing their birthdate, annual income, and government ID number to enable a credit score, they can generate zero-knowledge proofs that they are over 21, that their gross income on last year’s taxes exceeded a certain threshold, that they possess a valid government ID number, and that their credit score exceeded a certain threshold within the past week.

Strong, distributed ledger-based identity establishes a global source of truth in this use case, and this delivers value to many parties. Applicants can give consent, and everyone can agree on when and with what conditions it was given; lenders can demonstrate equal opportunity conformance with an immutable audit trail; the marketplace operates more efficiently.

This use case only gets more compelling when the goodness of other Hyperledger projects is added. Hyperledger Burrow could turn loan applications into contracts, and attach them to strong identities as a seamless next step. A membership system based on Hyperledger Fabric could link to the pre-existing and self-sovereign identity of the application, allowing them to service loans and be a responsible customer of the lender without painful onboarding. And so forth.
