\subsection{Intro}
The primary drivers for adoption of blockchain in financial service industry are considerations for privacy, confidentiality, and accountability. Compliance guidelines like “Anti Money Laundering” and “Know Your Customer” demand that users/customers are known and have been given clearance by their bank and/or the market infrastructure provider. These requirements drive the adoption of primarily permissioned and private blockchains. Public blockchains still carry the risk of compromising participants' confidentiality and privacy. These considerations together with very large volumes of transactions are the primary reasons that private consortium blockchains are gaining momentum in adoption of distributed ledger technology.

Among various use cases in financial services, and especially in capital markets, post trade activities is the prime area, which can benefit from adoption of blockchain.

\subsection{Post trade activities overview}
In trading post trade processing comprises all activities after the closure of a transaction. This general description is valid for all types of trading, OTC (over-the-counter) trading as well as trades executed at exchanges. OTC trading takes place when the trade counterparties interact directly or via brokerage services.

On a high level post-trade-processing comprises of the following operational steps
\begin{description}
\item [Trade validation] - activities taking place directly after trade execution, mainly validation and confirmation of the actual trading activities amongst the trade participants or through exchange. 
\item [Clearing] - alignment and  matching of the actual trade instructions and confirmations across the different counterparties as well as potential netting activities. In case the counterparties have agreed on bilateral margining or the transactions are cleared through a clearing house the counterparty/settlement risk arising between the time of concluding the trade and the time of settlement (typically 2 - 3 days) is mitigated. 
\item [Settlement] - the (legal) realization of the actual contractual obligations to reach the finality of the transaction. This includes support processes like the notification of all relevant entities affected by the transaction.
\item [Custody activities] - custodians are responsible for the safekeeping of securities. As such the positions held by the trade counterparties have to be adjusted. 
\end{description}

Besides these operational steps, post-trade-processing typically contains reporting requirements regarding the business transaction under consideration. Amongst these are counterparty internal risk reporting\footnote{The contribution of the transaction to the market and credit risk of the respective counterparts} and regulatory reporting. 

The operational steps as well as the reporting activities are in today’s setup typically a fragmented process chain spanning across a variety of departments of the respective counterparties, typically spread across a variety of entities, such as trade counterparties, brokers, settlement agents, central security depositories, clearing houses, thus resulting in a variety of interfaces. This consequently can result in a variety of reconciliation efforts along the process chain, between the trade counterparties as well as other entities/service providers involved, introducing inefficiencies in post trade processing.
 
\subsection{Efficiency benefits}
Implementing post-trade-processing on blockchain is bound to lead to efficiency gains as compared to the current implementation model. When settling via a blockchain system one could exploit the peer-to-peer property of a blockchain, i. e. one counterparty would insert the transaction details into the system and the other counterparty would verify and confirm.  Thus the confirmation processes would be processed within one system, thus rendering at separate confirmation processes obsolete and therefore contributing greatly to an overall process efficiency. In today’s world both parties would independently send their settlement instructions to a trusted 3rd party – the settlement agent – and this 3rd party would match both data sets and further process the settlement. Any mismatches in the initial instructions would lead to reconciliation efforts or even failed trades.

\subsection{Elimination of reconciliation}
The complexity of the multi-party interactions/interfaces is additionally reduced as all data from all from all process steps and actors resides on the blockchain and is accessible on a need-to-know basis. Therefore, the reconciliation processes should become obsolete all together. Also the reconciled data basis present within a blockchain based system could serve as an efficient basis for reporting activities, e. g. regulatory transaction and trade reporting.

\subsection{Drawbacks with elimination of netting}
Based on current market processes, settlement is taking place \emph{T+x}. This means transactions are settled at the end of day \emph{x} days after concluding the trade. This procedure brings certain advantages to the counterparties: as all transactions that have accumulated over a business day are settled end-of-day, existing netting agreements between counterparties (off-setting) can be applied and thus leading to capital savings. For example, let’s assume counterparty A and B conclude 2 transactions that are settled at the same day: A buys 100 pieces of stock X for 1000 USD from B and B buys 50 pieces of stock Y for 2000 USD from A. If both trades settle end-of-day B has to transfer 100 pieces of stock X and 1000 USD to A and A transfers 50 pieces Y to B. If these trades would settle near-time on a blockchain system the two individual transactions would settle separately and both counterparties would – without netting – need to provide more liquidity: A would need to transfer 1000 USD to B upon settlement of the first transaction and B would need to transfer the 2000 USD to A at a later stage, leading to higher liquidity requirements.

In summary moving towards a near-time settlement would reduce the potential for liquidity saving features like netting and thus potentially increase the capital needs for trading. Therefore, it is not clear if exploiting this blockchain feature would really be seen solely as an advantage by trading counterparties if used for settlement. Other aspects of post trade activities, such as trade validation and clearing, may still benefit from the immediate finality of the records. 

\subsection{Hyperledger projects for post trade}
\textbf{Hyperledger Fabric} channels provide an excellent solution to the problems of privacy and confidentiality. Ability to restrict data replication to only permissioned parties brings the benefits of the blockchain got data integrity and non-repudiation of transactions without compromising the security of the data. Additionally, reporting requirements - both internal and external - can be satisfied by including a regulatory agency and other oversight entities in the channel.

Fabric's smart contracts, chaincodes, can be further leverage to provide real-time reporting and, even, potentially risk calculations. The ability to call other chaincodes allows applications both to increase the level of confidence in the data by anchoring\footnote{Recording in an independent immutable public store} it on a different channel and provide consolidated risk view of the market reducing systemic risk.

\textbf{Hyperledger Indy} provides an ability to have unlinkable verifiable claims, which can be leveraged to report outstanding risk on a shared ledger without compromising the identity of the firm, but still allow a regulatory body to have a holistic view of the market, enabling it to prevent potential market crashes and major defaults.
