Security considerations for financial service applications
Irrespective of the use case considered for the financial service industry accountability and confidentiality considerations are prime drivers for the blockchain employed. Compliance guidelines like “Anti Money Laundering” and “Know Your Customer” demand that users/customers are known and have been given clearance by their bank and/or the market infrastructure provider. Therefore, it is safe to assume that all regulated financial service applications will be based on permissioned or private blockchains. Besides this user accountability user anonymity and transaction confidentiality are, depending on the use case, important criteria. Therefore, it is important for blockchains to be used as basis for financial service applications to allow the implementation of flexible role/access concepts allowing for user anonymity, confidentiality of the transaction content as well as ensuring that different transactions of the same user cannot be linked by observing the data flows. 
Post-trade processing use case
In trading post-trade-processing comprises all activities after the closure of a transaction. This general description is valid for all types of trading, OTC (over-the-counter)-trading as well as trades concluded at exchanges. OTC-trading takes place when the trade counterparties interact directly or via brokerage services.
On a high level post-trade-processing comprises of the following operational steps 
Trade validation – activities taking place directly after trade execution, mainly validation and confirmation of the actual trading activities amongst the trade participants or through exchangese trade 
Clearing – alignment and  matching of the actual trade instructions/ confirmations across the different counterparties as well as potential netting activities. In case the counterparties have agreed on bilateral margining or a the transactions are cleared through a clearing house the counterparty/settlemenmt risk arising between the time of concluding the trade  and the time of settlement (typically 2 - 3 days) is mitigated. 
Settlement – the (legal) realization of the actual contractual obligations (legal implementation of the outcome of a trade)., thus during settlement the finality of the transaction is reached. This includes support processes like the notification of all relevant entities affected by the transaction
Custody activities – custodians are responsible for the safekeeping of securities. As such the positions held by the trade counterparties have to be adjusted. 

Besides these operational steps post-trade-processing typically contains reporting requirements regarding the trade/transaction under consideration. Amongst these are counterparty internal risk reporting – the contribution of the transaction to the market and credit risk of the respective counterparts – and regulatory reporting. 

The operational steps as well as the reporting activities are in today’s setup typically a fragmented process chain spanning across a variety of departments of the respective counterparties, if not even spread across a variety of entities, e. g. trade counterparties, brokers, settlement agents, central security depositories, clearing houses, thus resulting in a variety of interfaces. This consequently can result in a variety of reconciliation efforts along the process chain, between the trade counterparties as well as other entities/service providers involved, introducing inefficiencies in post-trade-processing. 
 
Implementing post-trade-processing on blockchain is bound to lead to efficiency gains as compared to the current implementation model. When settling via a blockchain system one could exploit the peer-to-peer property of a blockchain, i. e. one counterparty would insert the transaction details into the system and the other counterparty would verify and confirm.  Thus the confirmation processes would be processed within one system, thus rendering at separate confirmation processes obsolete and therefore contributing greatly to an overall process efficiency. In today’s world both parties would independently send their settlement instructions to a trusted 3rd party – the settlement agent – and this 3rd party would match both data sets and further process the settlement. Any mismatches in the initial instructions would lead to reconciliation efforts or even failed trades.

The complexity of the multi-party interactions/interfaces is additionally reduced as all data from all from all process steps and actors resides on the blockchain and is accessible on a need-to-know basis. Therefore, the reconciliation processes should become obsolete all together. Also the reconciled data basis present within a blockchain based system could serve as an efficient basis for reporting activities, e. g.  regulatory transaction and trade reporting.
 
Based on current market usances settlement is taking place “T+x”, that means transactions are settled at the end of day on the x days after concluding the trade. This procedure brings certain advantages to the counterparties: As all transactions that have accumulated over a business day are settled end-of-day, existing netting agreements between counterparties (off-setting) can be applied and thus leading to capital savings. For example, let’s assume counterparty A and B conclude 2 transactions that are settled at the same day: A buys 100 pieces of stock X for 1000 USD from B and B buys 50 pieces of stock Y for 2000 USD from A. If both trades settle end-of-day B has to transfer 100 pieces of stock X and 1000 USD to A and A transfers 50 pieces Y to B. If these trades would settle near-time on a blockchain system the two individual transactions would settle separately and both counterparties would – without netting – need to provide more liquidity: A would need to transfer 1000 USD to B upon settlement of the first transaction and B would need to transfer the 2000 USD to A at a later stage, leading to higher liquidity requirements. In summary moving towards a near-time settlement would reduce the potential for liquidity saving features like netting and thus potentially increase the capital needs for trading. Therefore, it is not clear if exploiting this blockchain feature would really be seen solely as an advantage by trading counterparties. 